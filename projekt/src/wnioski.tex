\section{Wnioski}\label{sec:wnioski}
Modele do klasyfikacji danych można trenować na różne sposoby. W tym badaniu porównano dwa modele: sieć neuronową i k - NN.
Sieć neuronowa jest modelem, który wymaga uczenia - jest więc bardziej złożona obliczeniowo, wymaga więcej czasu na trenowanie i testowanie,
natomiast sposób w jaki jest to robione jest bardziej uniwersalny, elastyczny. Sieć neuronowa może być użyta do różnych problemów, nie tylko klasyfikacji.
K - NN jest modelem, który nie wymaga uczenia - jest więc mniej złożony obliczeniowo, wymaga mniej czasu na trenowanie i testowanie,
brakuje tu jednak elastyczności. K - NN może być użyty tylko do problemów klasyfikacji. Jednak używanie wiertarki do wbicia gwoździa nie jest najlepszym pomysłem.
stąd też zależy od problemu, który model wybrać. W tym badaniu KNN dała lepsze wyniki, ale nie oznacza to, że zawsze tak będzie. W zależności od problemu,
danych, parametrów, sieć neuronowa może dać lepsze wyniki. Warto więc znać oba modele i wiedzieć, kiedy który wybrać.