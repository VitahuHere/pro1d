\begin{abstract}
    Głównym zadaniem tego projektu jest stworzenie modelu do klasyfikacji spośród 26-ciu wielkich liter z alfabetu angielskiego.
    W ramach realizacji tego projektu użyte zostały mechanizmy uczenia maszynowego, przetwarzania danych i testowania klasyfikatora.
\end{abstract}

\section{Wstęp}\label{sec:wstep}
\subsection{Zbiór danych}\label{subsec:zbiordanych}
Dane pochodzą z bezpłatnego \href{https://archive.ics.uci.edu/dataset/59/letter+recognition}{repozytorium} Donald Bren school of Information and Computer Sciences (UCI ICS).
Badacze stworzyli ten zestaw z czarno-białych, prostokątnych obrazków liter zawierających jedną z 26 wielkich liter alfabetu angielskiego.
Bazowano na 20 różnych czcionkach, gdzie każdą literę z tych 20 czcionek poddano losowej dystorsji otrzymując 20 tysięcy unikatowych rekordów.
Każdy zawiera 16 kolumn. Pierwsza zawiera informację jaka to litera czyli klasa/kategoria, a pozostałe 15 zawierają informację o pikselach obrazu. 
