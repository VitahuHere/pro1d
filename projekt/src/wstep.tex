\section{Wstęp}\label{sec:wstep}

\subsection{Cel}\label{subsec:cel}
Celem tego badania jest porównanie klasyfikatorów k - NN i sieci neuronowej w zadaniu klasyfikacji liter alfabetu angielskiego.
W tym celu należy przeprowadzić analizę zbioru danych, przygotować go do uczenia maszynowego, a następnie przetestować oba klasyfikatory i porównać ich wyniki.

\subsection{Zbiór danych}\label{subsec:zbiordanych}
Dane pochodzą z bezpłatnego repozytorium \cite{misc_letter_recognition_59}
Donald Bren school of Information and Computer Sciences (UCI ICS) z grudnia 1990 roku.
Badacze stworzyli ten zestaw z czarno-białych, prostokątnych obrazków liter zawierających jedną z 26 wielkich liter alfabetu angielskiego.
Bazowano na 20 różnych czcionkach, gdzie każdą literę z tych 20 czcionek poddano losowej dystorsji otrzymując 20 tysięcy 
unikatowych\footnote{Rzeczywista liczba unikatowych wartości będzie przedstawiona później} rekordów.
Każdy zawiera 17 atrybutów. Pierwszy, zawiera informację jaka to litera, czyli klasa/kategoria, 
a pozostałe 16 zawierają informację o pikselach obrazu przedstawioną jako wartości całkowite od 0 do 15 z obu stron włącznie.
Żadna z kolumn nie ma brakującej wartości. Spośród informacji opisujących daną literę można wyróżnić pionową i poziomą pozycję ramki, 
szerokość i wysokość obrazka, łączną liczbę pikseli i różne operacje statystyczne
takie jak mediana, wariancja i korelacja danych \textit{x} oraz \textit{y}.

\subsection{Klasyfikatory}\label{subsec:klasyfikatory}
Do rozwiązania tego problemu można wykorzystać wiele algorytmów uczenia maszynowego, gdzie z najpopularniejszych można wymienić:
\begin{itemize}
    \item k - NN
    \item sieć neuronowa
    \item klasyfikator Bayesowski
\end{itemize}
W tym badaniu zaimplementowane i przetestowane zostały k - NN oraz sieć neuronowa.

\subsection{Użyte narzędzia i biblioteki}\label{subsec:narzedziaibiblioteki}
Istnieje wiele technologii umożliwiających implementację algorytmów uczenia maszynowego.
Jeśli implementację wszystkich elementów i algorytmów wykonuje się samemu, to język programowania ma duże znaczenie.
W przypadku bardziej zaawansowanych modeli, takich jak sieci neuronowe z większą liczbą ukrytych warstw, sieci konwolucyjne, bądź modele regresyjne,
należy użyć języków programowania o wysokiej wydajności, takich jak C++ lub Java. W tym badaniu użyto języka Python, który jest językiem wysokiego poziomu.
Oznacza to, że pomiędzy kodem napisanym w języku Python, a kodem maszynowym, jest więcej warstw abstrakcji, co powoduje, że jest on mniej wydajny.
Biblioteki takie jak Tensorflow, bądź PyTorch są napisane w języku C++, a interfejsy do nich są napisane w języku Python. Dzięki temu można w łatwy sposób korzystać z zaawansowanych modeli,
wykorzystując wydajność niskopoziomowego języka oraz prostotę języka wysokopoziomowego.

Jedną z najpopularniejszych i najprostszych bibliotek do uczenia maszynowego w języku Python jest scikit-learn\cite{scikit-learn} . Nie jest ona najszybsza, ani nie jest najbardziej rozbudowana,
ale jest łatwa w użyciu i posiada wiele przydatnych funkcji. Zawiera między innymi algorytmy wymienione w sekcji \ref{subsec:klasyfikatory}. Ze względu na relatywnie niewielki rozmiar zbioru danych,
wydajność nie jest najważniejszym kryterium. W tym badaniu wykorzystane zostały klasy KNeighborsClassifier i MLPClassifier, które implementują odpowiednio k - NN i sieć neuronową.

\subsection{Miary oceny}\label{subsec:miaryoceny}
Głównym kryterium oceny klasyfikatora jest jego trafność, czyli stosunek poprawnie sklasyfikowanych rekordów do wszystkich rekordów.
Nie jest to jednak jedyny i najlepszy sposób oceny klasyfikatora. W przypadku niezbalansowanych zbiorów danych, gdzie jedna klasa występuje znacznie częściej niż pozostałe,
należy wykorzystać metody biorące ten fakt pod uwagę. Jedną z takich metod to F-miara, czyli stosunek poprawnie sklasyfikowanych rekordów danej klasy do wszystkich rekordów sklasyfikowanych
jako ta klasa oraz stosunku poprawnie sklasyfikowanych rekordów danej klasy do wszystkich rekordów tej klasy. Wartość 1 oznacza idealną trafność klasyfikacji - nawet tej niezbalansowanej, 
analogicznie 0 oznacza zupełny brak trafności. W tym badaniu skupiono się na trafności klasyfikatorów, natomiast zmierzono również ich czasy uczenia i predykcji oraz F-miarę.