\begin{abstract}
    Głównym zadaniem tego projektu jest stworzenie modelu do klasyfikacji spośród 26-ciu wielkich liter z alfabetu angielskiego.
    W ramach realizacji tego projektu użyte zostały mechanizmy uczenia maszynowego, przetwarzania danych i oceny klasyfikatora.
\end{abstract}

\section{Wstęp}\label{sec:wstep}

\subsection{Zbiór danych}\label{subsec:zbiordanych}
Dane pochodzą z bezpłatnego repozytorium \cite{misc_letter_recognition_59} Donald Bren school of Information and Computer Sciences (UCI ICS) z grudnia 1990 roku.
Badacze stworzyli ten zestaw z czarno-białych, prostokątnych obrazków liter zawierających jedną z 26 wielkich liter alfabetu angielskiego.
Bazowano na 20 różnych czcionkach, gdzie każdą literę z tych 20 czcionek poddano losowej dystorsji otrzymując 20 tysięcy 
unikatowych\footnote{W teorii. Rzeczywista liczba unikatowych wartości będzie przedstawiona później} rekordów.
Każdy zawiera 17 atrybutów. Pierwszy, zawiera informację jaka to litera, czyli klasa/kategoria, a pozostałe 16 zawierają informację o pikselach obrazu przedstawioną jako wartości całkowite od 0 do 15 z obu stron włącznie.
Żadna z kolumn nie ma brakującej wartości. Spośród informacji opisujących daną literę można wyróżnić pionową i poziomą pozycję ramki, szerokość i wysokość obrazka, łączną liczbę pikseli i różne operacje statystyczne
takie jak mediana, wariancja i korelacja danych \textit{x} oraz \textit{y}.

\subsection{Klasyfikatory}\label{subsec:klasyfikatory}
Do rozwiązania tego problemu można wykorzystać wiele algorytmów uczenia maszynowego, gdzie z najpopularniejszych można wymienić:
\begin{itemize}
    \item k - NN
    \item sieć neuronowa
    \item klasyfikator Bayesowski
\end{itemize}
W tym badaniu zaimplementowane i przetestowane zostały k - NN oraz sieć neuronowa.

\subsection{Rezultaty}
