\section{Dobór hiperparametrów}\label{sec:dobor_hiperparametrow}
\subsection{Sieć neuronowa}\label{subsec:dobor_hiperparametrow_siec_neuronowa}
\subsubsection{Wpływ liczby neuronów w warstwach ukrytych}\label{subsubsec:liczba_neuronow}
Sieci neuronowe są mocno specyficznym modelem uczenia maszynowego, ponieważ nie ma jednoznacznej metody na dobór hiperparametrów.
Liczba neuronów, liczba warstw, funkcje aktywacji, funkcje straty, funkcje optymalizujące, współczynnik uczenia, liczba epok - wszystkie te parametry mają wpływ na wynik klasyfikacji.
Nie ma jednej metody na dobór hiperparametrów, więc trzeba przetestować różne kombinacje i wybrać tę, która daje najlepsze wyniki.
Zanim zdecydowano się na ostateczną architekturę sieci neuronowej, przetestowano różne kombinacje liczby neuronów w warstwach ukrytych:
\begin{itemize}
    \item 10, 10
    \item 16, 32
    \item 32, 64
    \item 64, 128
    \item 128, 256
    \item 256, 512,
\end{itemize}
jak i również odwrotność kolejności tych kombinacji. Mniejsze liczby neuronów, czyli poniżej 32 zatrzymywały trenowanie po około 20 epokach, ponieważ nie było poprawy wyników, 
które i tak nie były wysokie.
Najlepsze wyniki uzyskano dla 128 i 256 neuronów w warstwach ukrytych.
\subsubsection{Wpływ liczby epok na wynik klasyfikacji}\label{subsubsec:liczba_epok}
Liczba epok to hiperparametr, który określa, ile razy sieć neuronowa przejdzie przez cały zbiór danych. Wartość tego parametru bezpośrednio wpływa na jakość modelu.
Zbyt mała liczba epok może spowodować niedouczenie, a zbyt duża - przeuczenie. Należy monitorować wyniki klasyfikacji w trakcie trenowania i zatrzymać je, gdy nie ma poprawy, bądź ta poprawa jest zbyt mała.
Przetestowaniu poddano następujące liczby epok: 10, 20, 30, 40, 50, 60, 70, 80, 90, 100.
Optymalną wartością okazała się być 40.
\subsection{K - NN}\label{subsec:dobor_hiperparametrow_knn}
\subsubsection{Wpływ liczby sąsiadów}\label{subsubsec:liczba_sasiadow}
K - NN jest algorytmem, który nie wymaga uczenia, więc jedyny hiperparametr jaki można dobierać to liczba sąsiadów.
Do tego celu posłużono się klasą GridSearchCV z biblioteki scikit-learn.
Pod uwagę wzięto następujące liczby sąsiadów: 3, 5, 7, 9, 11, 13, 15. Klasa GridSearchCV
przetestowała wszystkie kombinacje i wybrała tę, która dawała najlepsze wyniki - 3.